A lo largo del proyecto, han surgido múltiples dificultades. En primer lugar, el rendimiento de Android Studio y del emulador no es el esperado, lo que ha dificultado algo la tarea. Además, tuve problemas a la hora de gestionar los permisos y cómo se realizaba el conteo del tiempo, pues había que pulsar varias veces para que se registrase una entrada o una salida, y las notificaciones no siempre se enviaban o, peor aún, se enviaban cada segundo obligando a detener la aplicación, algo que fue corregido.

También ha resultado algo complicado implementar la lógica del inicio de sesión, principalmente porque la aplicación se creó siendo pensada para un solo usuario y, por tanto, ha sido necesario adaptar cada método para que funcione correctamente con múltiples usuarios, siendo necesario acceder y añadir un parámetro más en cada consulta. Tampoco ha sido sencillo almacenar las imágenes de perfil en el servidor debido a las limitaciones del mismo, aunque se ha solventado comprimiendo las imágenes y reduciendo su tamaño.

Además, ha resultado algo complicado gestionar el salto de una alarma programada a una hora exacta, principalmente por los permisos y porque en ocasiones se borran. El uso de FCM tampoco ha sido sencillo al haber modificado Google los métodos de acceso para Firebase, requiriendo usar su nueva autenticación. Por último, el \textit{content provider} del calendario tampoco ha sido fácil, debido a que es necesario acceder a las Cuentas para verificar que puede editarse un calendario para añadir un evento.