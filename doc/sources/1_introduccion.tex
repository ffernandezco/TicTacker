\textit{TicTacker} es una aplicación móvil para Android basada en Java + Gradle que permite ayudar a empresas y a autónomos a seguir un registro de su jornada laboral. Para ello, permite realizar un seguimiento completo de las horas y minutos trabajados cada día, permitiendo además configurar una jornada laboral con sus correspondientes horas y días laborales a fin de conocer las horas extraordinarias realizadas cada día y recibir notificaciones en caso de sobrepasar la jornada descrita. Permite a los empleados registrar su perfil para obtener una experiencia más personalizada y se sincroniza con un servidor PHP para almacenar toda la información en una base de datos MySQL remota. Es además posible personalizar su logotipo y el idioma para adaptarse a las necesidades de la empresa, así como generar ficheros CSV con los fichajes registrados que posteriormente puede importarse si así se desea o configurar recordatorios para fichar a tiempo.

\subsection{Instalación y configuración}

Para hacer uso de \textit{TikTacker}, puede utilizarse el fichero APK para realizar la instalación. Una vez instalada, será necesario autenticarse, para lo que es posible crear una nueva cuenta usando el asistente o bien utilizar el usuario de demostración haciendo uso de las siguientes credenciales:

\begin{itemize}
    \item \textbf{Usuario}: \texttt{demo}
    \item \textbf{Contraseña}: \texttt{demo}
\end{itemize}

\begin{tcolorbox}
    [colback=red!5!white,colframe=red!75!black,fonttitle=\bfseries,title=Conexión a Internet]
     Esta edición de \textit{TicTacker} realiza conexiones con el servidor de Internet \url{http://ec2-51-44-167-78.eu-west-3.compute.amazonaws.com/ffernandez032/WEB/}. Se recomienda verificar que se puede acceder a dicha dirección a través del navegador del dispositivo antes de comenzar a utilizar la aplicación.
    \end{tcolorbox}

A partir de entonces, podrán comenzar a utilizarse las funcionalidades de la aplicación.