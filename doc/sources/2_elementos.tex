Para construir la aplicación, se han hecho uso de las siguientes características y elementos propios de Android:

\begin{itemize}
    \item \textbf{Uso de \textit{RecyclerView} + \textit{CardView} para mostrar listados de elementos con diferentes características}: el historial de fichajes utiliza \textit{RecyclerView} para mostrar listas de fichajes en un formato vertical. Cada elemento de la lista se muestra en una vista personalizada (\textit{CardView}), mostrando información detallada de cada fichaje.
    \item \textbf{Uso de una base de datos local, para listar, añadir y modificar elementos y características de cada elemento}: tal y como se detallará más adelante, el proyecto hace uso de una base de datos SQLite para almacenar los datos de los fichajes, los usuarios y las configuraciones.
    \item \textbf{Uso de diálogos}: la aplicación cuenta con múltiples diálogos, incluyendo los detalles de fichajes, que permiten ver todos los datos o editarlos si es preciso, mensajes de confirmación a la hora de borrar información, o un mensaje que se muestra si aún no ha finalizado la jornada laboral pero se trata de marcar una salida.
    \item \textbf{Uso de notificaciones locales}: la aplicación utiliza notificaciones para alertar a los usuarios cuando han completado su tiempo de trabajo diario o cuando se llega a las horas extra.
    \item \textbf{Control de la pila de actividades}: se controla que la aplicación finalice las actividades que no se están usando para una gestión correcta de la memoria. Además, si se modifica algún parámetro de configuración como el idioma, se recarga la aplicación.
    \item \textbf{Uso de \textit{Fragments}}: la aplicación cuenta con 3 \textit{Fragments} diferentes: uno para el fichaje, otro para el historial y otro para la configuración, pudiéndose acceder a ellos directamente desde el menú y adaptándose a la orientación del dispositivo y el tamaño de la pantalla.
    \item \textbf{Aplicación multiidioma y añadir la opción de cambiar idioma en la propia aplicación}: la aplicación está disponible en inglés y castellano, permitiendo el cambio de idioma desde el apartado de Ajustes. Por defecto, usa el idioma del dispositivo.
    \item \textbf{Uso de ficheros de texto}: se permite exportar e importar los fichajes de un usuario en formato CSV, además de cambiar el logotipo que se muestra en el menú por medio de los ajustes, manipulando tanto ficheros de texto como imágenes.
    \item \textbf{Uso de Preferencias, para guardar las preferencias del usuario en cuanto a mostrar/esconder cierta información}: la aplicación permite controlar el idioma, el logotipo y definir una jornada laboral, marcando tanto los días laborables como las horas que se trabajan cada semana. En función de este dato, se cargará el tiempo restante de la jornada y se enviarán notificaciones.
    \item \textbf{Crear estilos y temas propios, para personalizar fondos, botones, etc.}: se ha hecho uso de un tema propio basado en Material Design de Google, en color verde a juego con el logotipo de la app.
    \item \textbf{Uso de intents implícitos para abrir otras aplicaciones, contactos, etc.}: al registrar un fichaje, se almacenan las coordenadas desde las que se ha realizado, permitiendo posteriormente en el detalle abrir la ubicación en una aplicación que lo permita y que esté instalada (por ejemplo, Google Maps).
    \item \textbf{Pantalla de login (y registro), para guardar credenciales de usuario en la base de datos local}: se ha incluido, además de una pantalla para iniciar sesión, una sección que permite dar de alta un nuevo usuario con su nombre y contraseña para permitir el acceso, pudiendo almacenar fichajes de varios empleados en un mismo dispositivo.
    \item \textbf{Añadir una barra de herramientas (ToolBar) personalizada en la aplicación así como un panel de navegación (Navigation Drawer)}: se ha incluido una barra de herramientas adaptada a los dispositivos modernos, así como un panel que permite moverse entre los 3 fragmentos creados para la app fácilmente.
\end{itemize}