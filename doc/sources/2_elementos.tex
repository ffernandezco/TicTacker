Esta aplicación es una continuación de la \href{https://github.com/ffernandezco/DAS-Proyecto}{primera versión entregada}. Para construirla, se han hecho uso de las siguientes características y elementos propios de Android, además de los detallados en la \href{https://github.com/ffernandezco/DAS-Proyecto/blob/main/doc/handout.pdf}{documentación de dicha versión}:

\begin{itemize}
    \item \textbf{Uso de una base de datos remota para el registro y la identificación de usuarios}: la base de datos anterior, además de haberse ampliado, se ha alojado en una base de datos remota a la que se accede desde ficheros PHP contenidos en el servidor \url{ec2-51-44-167-78.eu-west-3.compute.amazonaws.com}. Se permite registrar usuarios e iniciar sesión, se almacenan los fichajes, los datos de los perfiles y la configuración.
    \item \textbf{Integrar los servicios Google Maps u OpenStreetMap y Geolocalización en una actividad}: a la hora de registrar un fichaje, se almacena la ubicación en caso de estar disponible y se guarda en la base de datos. Accediendo al historial de fichajes y seleccionando uno existente, es posible ver, además de los detalles, un pequeño mapa de OpenStreetMap con \href{https://github.com/osmdroid/osmdroid}{OSMdroid} en el que se detalla la ubicación.
    \item \textbf{Uso de algún content provider para añadir, modificar o eliminar datos}: al guardar un fichaje, ahora se añade también al calendario propio del dispositivo para poder ver todos con mayor detalle. Además de añadir los eventos, en caso de no haber un calendario editable se encarga de crearlo en la cuenta de Google.
    \item \textbf{Captar imágenes desde la cámara, guardarlas en el servidor y mostrarlas en la aplicación}: en la sección de perfil, cada usuario puede elegir si tomar una foto de perfil haciendo uso de la cámara del dispositivo o si prefiere obtenerla desde la galería de imágenes del dispositivo. Para evitar los límites de PHP y saturar el servidor, las imágenes se almacenan en Base64 directamente en la base de datos, con una resolución máxima de 160x160 px. Después se muestran tanto en la vista del perfil como en el menú.
    \item \textbf{Implementación de un servicio en primer plano y gestión de mensajes broadcast durante el servicio}: al comenzar un fichaje, se registra en primer plano (\textit{Foreground}) y se muestra una notificación silenciosa persistente que se actualiza periódicamente para permitir ver el tiempo que ha pasado desde el fichaje. Deja de actualizarse al marcar la salida.
    \item \textbf{Uso de mensajería FCM}: la aplicación cuenta con la mensajería FCM de Google Firebase para permitir la recepción de notificaciones push. Se almacenan los diferentes tokens generados en la base de datos, y desde el servicio web \url{http://ec2-51-44-167-78.eu-west-3.compute.amazonaws.com/ffernandez032/WEB/notificar.php} es posible enviar una notificación a todos los usuarios\footnote{Se entiende que cada usuario tendrá únicamente un dispositivo, por lo que si se inicia sesión en un nuevo dispositivo se actualiza el token en la base de datos y, por tanto, el dispositivo anterior deja de recibir notificaciones. Se recomienda crear diferentes usuarios nuevos si se van a realizar pruebas con diferentes dispositivos.} de la aplicación.
    \item \textbf{Desarrollar un widget que tenga, al menos, un elemento que se actualice automáticamente de manera periódica}: se ha incluido un widget que permite consultar, de un vistazo, el estado del fichaje (con un botón para fichar o salir), así como las horas restantes en caso de tener un fichaje en curso.
    \item \textbf{Uso de algún servicio o tarea programada mediante alarma}: a través de la configuración, es posible habilitar un recordatorio para fichar. Este recordatorio configura una alarma y, en caso de que no se haya registrado ningún fichaje para la hora establecida, se envía una notificación programada. Para poder hacer uso de este servicio, será necesario habilitar los permisos correspondientes en la configuración del dispositivo.
\end{itemize}